	\paragraph{Aufgabe 1: Aditionstheoreme}
	\begin{itemize}
		\item[(a)] Zu zeigen ist, dass
		\begin{align*}
			\tan( x + y ) = \frac{ \tan( x ) + \tan( y ) }{ 1 - \tan( x ) \tan( y )}
		\end{align*}
		für alle $ x, y \in \mathbb{R} $, für die $ \tan( x + y ), \tan( x ) $ und $ \tan( y )$ definiert sind.
		\begin{align*}
			\tan( x + y ) 
			= \frac{ \sin( x + y ) }{ \cos( x + y ) } 
			= \frac{ \sin( x ) \cos( y ) + \cos( x ) \sin( y ) }
			{ \cos( x ) \cos( y ) - \sin( x ) \sin( y ) }
			\\
			= \frac{ \frac{ \sin( x ) \cancel{ \cos( y ) } }{ \cos( x ) \cancel{ \cos( y ) } } + \frac{ \cancel{ \cos( x ) } \sin( y ) }{ \cancel{ \cos( x ) } \cos( y ) } }
				{ \frac{ \cancel{ \cos( x )  \cos( y ) } }{ \cancel{ \cos( x ) \cos( y ) } } - \frac{ \sin( x ) \sin( y ) }{ \cos( x ) \cos( y ) } }
			= \frac{ \frac{ \sin( x )}{ \cos( x ) } + \frac{ \sin( y ) }{ \cos( y ) } }
				{ 1 - \frac{ \sin( x ) \sin( y ) }{ \cos( x ) \cos( y ) } } 
			= \frac{ \tan( x ) + \tan( y ) }{ 1 - \tan( x ) \tan( y ) }
		\end{align*}
		
		\item[(b)]
%		Zeigen Sie, dass folge folgende Aussage für alle $ n \in \N $ und $ x \in \mathbb{ R } $ gilt.
%		\begin{align*}
%			\cos ( nx ) &= \sum_{ j = 0 }^{ \lfloor \frac{ n }{ 2 } \rfloor } (-1)^j \binom{ n }{ 2j } \cos{ x }^{ n - 2j } \sin( x )^{ 2j }
%			\\
%			\sin ( nx ) &= \sum_{ j = 0 }^{ \lfloor \frac{ n - 1 }{ 2 } \rfloor } (-1)^j \binom{ n }{ 2j + 1 } \cos{ x }^{ n - 2j - 1 } \sin( x )^{ 2j + 1 }
%		\end{align*}
%		(IS)
%		\begin{align*}
%			\cos( (n+1) x ) = \cos(nx + x )
%			= \cos( nx ) \cos( x ) + \sin( nx ) \sin( x )
%			\\
%			= \sum_{ j = 0 }^{ \lfloor \frac{ n }{ 2 } \rfloor } (-1)^j \binom{ n }{ 2j } \cos{ x }^{ n - 2j + 1} \sin( x )^{ 2j }
%			+ \sum_{ j = 0 }^{ \lfloor \frac{ n - 1 }{ 2 } \rfloor } (-1)^j \binom{ n }{ 2j + 1 } \cos{ x }^{ n - 2j - 1 } \sin( x )^{ 2j + 2 }
%			\\ \text{ 1. Fall } n \text{ gerade: } \lfloor \frac{ n + 1 }{ 2 } \rfloor = \lfloor \frac{ n }{ 2 } \rfloor = \frac{ n }{ 2 }, 
%			\lfloor \frac{ n - 1 }{ 2 } \rfloor = \lfloor \frac{ n }{ 2 } \rfloor - 1
%			\\
%			= \sum_{ j = 0 }^{ \frac{ n }{ 2 } } (-1)^j \binom{ n }{ 2j } \cos{ x }^{ n - 2j + 1} \sin( x )^{ 2j }
%			+ \sum_{ j = 0 }^{ \frac{ n }{ 2 } - 1} (-1)^j \binom{ n }{ 2j + 1 } \cos{ x }^{ n - 2j - 1 } \sin( x )^{ 2j + 2 }
%			\\
%			= (-1)^{n/2} \binom{ n }{ n } \cos{ x }^{ n - n + 1} \sin( x )^{ n } +
%			\\ \sum_{ j = 0 }^{ \frac{ n }{ 2 } - 1 } (-1)^j \binom{ n }{ 2j } \cos{ x }^{ n - 2j + 1} \sin( x )^{ 2j }
%			+ (-1)^j \binom{ n }{ 2j + 1 } \cos{ x }^{ n - 2j - 1 } \sin( x )^{ 2j + 2 }
%			\\
%			= (-1)^{n/2} \cdot 1 \cdot \cos( x ) \sin( x )^n +
%			\\
%			\sum_{ j = 0 }^{ \frac{ n }{ 2 } - 1 } (-1)^j \cos( x )^{ n - 2j - 1} \sin( x )^{ 2j }
%			\left( \binom{ n }{ 2j }  \right)
%			\\
%			= \dots
%			= \sum_{ j = 0 }^{ \lfloor \frac{ n + 1 }{ 2 } \rfloor } (-1)^j \binom{ n + 1 }{ 2j } \cos{ x }^{ n - 2j + 2} \sin( x )^{ 2j }
%		\end{align*}
	

	\end{itemize}


	\paragraph{Aufgabe 2: Stetgkeit I }
	\begin{itemize}
		\item[(a)]
		Sei $ w \in \mathbb{ C } $ und sei $ f : \mathbb{ C } \mapsto \mathbb{ R } $ definiert durch
		\begin{align*}
			f(z) = \vert w - z \vert
		\end{align*}
		Zeigen Sie, dass $ f $ in jedem $ z \in \mathbb{ C } $ stetig ist.
		
		Sei $ z = x_1 + i y_ 1 $ und $ w = x_2 + i y_2 $.
		Mit $ x_1, x_2, y_1, y_2 \in \mathbb{ R } $.
		\begin{align*}
			\vert w - z \vert = \vert x_2 + i y_2 -  ( x_1 + i y_ 1 ) \vert = \vert x_2 - x_1 + i (y_2 - y_1) \vert
			= ( \underbrace{ (x_2 - x_1)^2  + (y_2 - y_1)^2 }_{\in \mathbb{ R }^+_0 } )^{1/2}
		\end{align*}
		Da die Potenzfunktion auf ganz $ \mathbb{ R } $ stetig ist und die Wurzelfunktion auf $ \mathbb{ R }^+_0 $ ebenfalls, folgt, dass auch $ f(z) $ stetig ist.
		
		
		\item[(b)] Betrachten Sie die Funktion $ X_{ \mathbb{ Q } } : \mathbb{ R } \mapsto \mathbb{ R }$
		\begin{align*}
			X_{ \mathbb{ Q } } =
			\left\lbrace  \begin{matrix}
				1 & x \in \mathbb{ Q }
				\\
				0 & x \in \mathbb{ R } \setminus \mathbb{ Q }
			\end{matrix} \right.
		\end{align*}
		Zeigen Sie, dass $ X_{ \mathbb{ Q } } $ in keinem $ x \in \mathbb{ R } $ stetig ist.

		Sei $ r \in \mathbb{ R } \setminus \mathbb{ Q } $.
		Wie in der Vorlesung gezeigt, lässt sich jede reele Zahl als b-adischer Bruch darstellen.
		D. h. es existieren $ m \in \NN_0, a_{-m}, ...,  a_0, a_1, ... \in \mathbb{ Q } $ mit
		\begin{align*}
			r = \sum_{ k = -m }^\infty a_k 10^{-k}
		\end{align*}
		Wir defiiern die Folge $ b_n $ mit
		\begin{align*}
			b_n = \sum_{ k = -m }^n a_k 10^{-k}
		\end{align*}
		Die Folge $ b_n $ konvergiert gegen $ r $.
		Da $ a_k $ und $ 10^-k \in \mathbb{ Q } $ für alle $ k \in \mathbb{ Z } $ gilt,
		 $ X_{ \mathbb{ Q } }( b_n ) = 1 $ für alle $ n \in \mathbb{ Z } $.
		Es gilt daher $ X_{ \mathbb{ Q } }( b_n ) \nrightarrow X_{ \mathbb{ Q } }( r ) $.

		Sei $ q \in \mathbb{ Q } $ und $ (d_n)_{ n \in \NN } $ eine Folge wie folgt defineirt.
		\begin{align*}
			d_n = q + \frac{ \sqrt{ 2 } }{ n }
		\end{align*}
		Die Folge $ (d_n)_{ n \in \NN } $ konvergiert gegen $ q $ ( Beweis: Additivität der Limiten).
		Für alle $ n \in \NN $ gilt, da $ \sqrt{ 2 } \in \mathbb{ R } $, $ d_n \in \mathbb{ R } $.
		Es gilt also $ 0 =  X_{ \mathbb{ Q } }( d_n ) \nrightarrow X_{ \mathbb{ Q } }( q ) = 1 $


		\item[(c)] 
		Seien $ f, g : \mathbb{ R } \mapsto \mathbb{ R } $ stetige Funktionen und $ f = g $ auf $ \mathbb{ Q } $.
		Zeigen Sie, dass dann schon $ f = g $ auf $ \mathbb{ R } $ gilt.

		Sei $ r \in \mathbb{ R } $.
		Wir stellen $ r $ als 10-adischen Bruch mit $ m \in \NN_0 $ und $ a_{-m}, .., a_0, a_1, ... \in \mathbb{ Q } $ wie folg da.
		\begin{align*}
			r = \sum_{ k = -m }^\infty a_k 10^{-k}
		\end{align*}
		Wähle $ (b_n)_{n \in \NN} $ als Folge mit $ b_n = \sum_{ k = -m }^n a_k 10^{-k} $.
		Da $ b_n \in \mathbb{ Q } $ für alle $ n \in \NN $ gilt, $f( b_n ) = g ( b_n ) $ für alle $ n \in \NN $.
		Da $ f $ und $ g $ stetige Funktion auf $ \mathbb{ R } $ sind gilt, 
		\begin{align*}
			f( b_ n ) \mapsto f( r ) \\
			g( b_ n ) \mapsto g( r )
		\end{align*}
		Da alle Folgenglider der Folgen $ f( b_n ) $ und $ g( d_n ) $ gleich sind und beide konvergieren, konvergieren sie gegen den gelichen Grenzwert.

	\end{itemize}


	\paragraph*{Aufgabe 3: Stetigkeit II}
	Betrachten Sie die folgenden Funktionen und bestimmen Sie, für welche $ x $ die Funktionen stetig sind und beweisen Sie Ihre Aussage.
	\begin{itemize}
		\item[(a)]
		Sei $ f : \mathbb{ R } \mapsto \mathbb{ R } $ definiert durch $ f( x ) = x^2 $.
		
		Sei id $ _\mathbb{ R }$ die Identiätsfunktion auf $ \mathbb{ R } $.
		Diese ist trivialerweise stetig.
		Die Funktion $ f $ lässt sich wie folgt als Multiplikation der Identiätsfunktion darstellen: $ f = \text{id} _\mathbb{ R } \cdot \text{id} _\mathbb{ R } $.
		Nach dem Satz aus der Vorlesung gilt, dass das Produkt zweier stetiger Funktionen ebenfalle steig ist.
		Es gilt daher, dass $ f $ stetig ist.

		\item[(b)]
		Sei $ g : \mathbb{ R } \mapsto \mathbb{ R } $ definiert durch
		\begin{align*}
			g( x ) =
			\left\lbrace \begin{matrix}
				\sqrt{ x } & \text{ für } x > 1
				\\
				\sqrt{ 1 - x } & \text{ für } x \leq 1
			\end{matrix}\right.
		\end{align*}
		
		$ f $ ist stetig für alle $ x \in \mathbb{ R } \setminus \lbrace 1 \rbrace $, da Wurzel und Summen steig sind, sowie auch Verknüpfung stetiger Funktionen.
		In $ x = 1 $ ist $ f( x ) $ nicht stetig.
		\\ 
		Beweis.
		Sei $ ( a_n )_{n \in \NN } $ eine Folge mit $ a_n = 1 + \tfrac{ 1 }{ n } $.
		Es gitl also $ a_n \mapsto 1 , n \mapsto \infty $.
		\begin{align*}
			f( a_n ) = \sqrt{ 1 + \frac{ 1 }{ n } } \mapsto 1 \neq  0 = f( 1 )
		\end{align*}


		\item[(c)]
		Sei $ g : \mathbb{ R }_{> 0} \mapsto \mathbb{ R } $ definiert durch $ h( x ) = \sqrt{ x } \cdot \sin( \frac{ 1 }{ x } ) $.

		Die Wurzelfunktion ist auf ganz $ \mathbb{ R }^+ $ stetig, wie in der Vorlesung gezeigt.
		Die Sinusfunktion von $ \frac{ 1 }{ x } $ ist stetig in allen $ x \in \mathbb{ R }^+ $, außer für $ x $ gegen 0.
		Wir müssen uns allso nur noch die Steitkeit in Null anschauen.
		Wir möchten mit dem delta-epsilon-Kriterium die Stetigketi von $ h $ in Null zeigen.
		D. h. es soll gelten:
		\begin{align*}
			\forall \varepsilon > 0 : \exists \delta > 0 : \forall x \in \mathbb{ R }_{> 0} \text{ mit }
			\vert x - 0 \vert < \delta \Rightarrow \vert h( x ) -  h( 0 ) \vert < \varepsilon
		\end{align*}
		Wähle $ \delta  < \varepsilon^2 $.
		Wir betrachtel also alle $ x \in \mathbb{ R }_{> 0} $ mit $ x < \sqrt{ \varepsilon} $.
		Wir definiiern 
		\begin{align*}
			\vert h( x ) -  h( 0 ) \vert = \vert h( x ) \vert = \left\vert \sqrt{ x } \cdot \sin( \frac{ 1 }{ x }) \right\vert
			= \sqrt{ x } \ \left\vert \sin( \frac{ 1 }{ x }) \right\vert \overset{| \sin | \leq 1} \leq \sqrt{ x } < \sqrt{ \delta } < \varepsilon
		\end{align*}

	\end{itemize}
