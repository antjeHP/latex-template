	\paragraph{Aufgabe 1: Aditionstheoreme}
	\begin{itemize}
		\item[(a)] Zu zeigen ist, dass
		\begin{align*}
			\tan( x + y ) = \frac{ \tan( x ) + \tan( y ) }{ 1 - \tan( x ) \tan( y )}
		\end{align*}
		für alle $ x, y \in \mathbb{R} $, für die $ \tan( x + y ), \tan( x ) $ und $ \tan( y )$ definiert sind.
		\begin{align*}
			\tan( x + y ) 
			= \frac{ \sin( x + y ) }{ \cos( x + y ) } 
			= \frac{ \sin( x ) \cos( y ) + \cos( x ) \sin( y ) }
			{ \cos( x ) \cos( y ) - \sin( x ) \sin( y ) }
			\\
			= \frac{ \frac{ \sin( x ) \cancel{ \cos( y ) } }{ \cos( x ) \cancel{ \cos( y ) } } + \frac{ \cancel{ \cos( x ) } \sin( y ) }{ \cancel{ \cos( x ) } \cos( y ) } }
				{ \frac{ \cancel{ \cos( x )  \cos( y ) } }{ \cancel{ \cos( x ) \cos( y ) } } - \frac{ \sin( x ) \sin( y ) }{ \cos( x ) \cos( y ) } }
			= \frac{ \frac{ \sin( x )}{ \cos( x ) } + \frac{ \sin( y ) }{ \cos( y ) } }
				{ 1 - \frac{ \sin( x ) \sin( y ) }{ \cos( x ) \cos( y ) } } 
			= \frac{ \tan( x ) + \tan( y ) }{ 1 - \tan( x ) \tan( y ) }
		\end{align*}	

	\end{itemize}
